\hypertarget{index_description}{}\section{Description}\label{index_description}
BFilt is a multi-platform and open-source C++ bayesian filtering library. It contains useful and classical algorithms in state estimation of hidden markov models. So you can easily construct discrete-disrete (DD) and continuous-discrete (CD) models (linear or nonlinear) for filtering (Kalman, EKF, UKF, particle filters, ...) and simulation methods. Indeed, markovian model simulators can be used for particle filters. Libraries such as BFL and Bayes++ consider only discrete-discrete filtering. With BFilt, you can easily construct your own CD or DD models for filtering. For CD models stochastic discretization methods (Euler, Runge Kutta, Local linearization, Heun) are implemented in simulation and filtering.\hypertarget{index_implementation}{}\section{Dependances}\label{index_implementation}
LAPACK and CPPLAPACK libraries are used for linear algebra operations. For best performances it is recommended to compile yourself the LAPPACK libraries with ATLAS. The Gnu Scientific Library (GSL) achieves random drawing in simulators. These open-source and multi-platform libraries must be installed before install BFilt.\hypertarget{index_Installation}{}\section{Installation}\label{index_Installation}
Go to the bin directory 

\begin{Code}\begin{verbatim} cd BFilt/bin
\end{verbatim}
\end{Code}

 Run Cmake ($>$2.6) 

\begin{Code}\begin{verbatim} cmake ../src
\end{verbatim}
\end{Code}

 Compile Bfilt 

\begin{Code}\begin{verbatim} make
\end{verbatim}
\end{Code}

 Install BFilt in /usr/local/lib or /usr/local/inlcude 

\begin{Code}\begin{verbatim} make install
\end{verbatim}
\end{Code}

 If you want to change the default install directory you can type 

\begin{Code}\begin{verbatim} ccmake 
\end{verbatim}
\end{Code}

 and change CMAKE\_\-INSTALL\_\-PREFIX \hypertarget{index_auteur}{}\section{Auteur}\label{index_auteur}
\begin{Desc}
\item[Author:]paul $<$\href{mailto:paul.frogerais@univ-rennes1.fr}{\tt paul.frogerais@univ-rennes1.fr}$>$ \end{Desc}
\begin{Desc}
\item[Date:]Fri Sep 12 18:34:36 2008 \end{Desc}
